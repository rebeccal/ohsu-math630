\documentclass[]{article}
\usepackage{lmodern}
\usepackage{amssymb,amsmath}
\usepackage{ifxetex,ifluatex}
\usepackage{fixltx2e} % provides \textsubscript
\ifnum 0\ifxetex 1\fi\ifluatex 1\fi=0 % if pdftex
  \usepackage[T1]{fontenc}
  \usepackage[utf8]{inputenc}
\else % if luatex or xelatex
  \ifxetex
    \usepackage{mathspec}
  \else
    \usepackage{fontspec}
  \fi
  \defaultfontfeatures{Ligatures=TeX,Scale=MatchLowercase}
\fi
% use upquote if available, for straight quotes in verbatim environments
\IfFileExists{upquote.sty}{\usepackage{upquote}}{}
% use microtype if available
\IfFileExists{microtype.sty}{%
\usepackage{microtype}
\UseMicrotypeSet[protrusion]{basicmath} % disable protrusion for tt fonts
}{}
\usepackage[margin=1in]{geometry}
\usepackage{hyperref}
\hypersetup{unicode=true,
            pdftitle={Midterm Exam - Key},
            pdfborder={0 0 0},
            breaklinks=true}
\urlstyle{same}  % don't use monospace font for urls
\usepackage{color}
\usepackage{fancyvrb}
\newcommand{\VerbBar}{|}
\newcommand{\VERB}{\Verb[commandchars=\\\{\}]}
\DefineVerbatimEnvironment{Highlighting}{Verbatim}{commandchars=\\\{\}}
% Add ',fontsize=\small' for more characters per line
\usepackage{framed}
\definecolor{shadecolor}{RGB}{248,248,248}
\newenvironment{Shaded}{\begin{snugshade}}{\end{snugshade}}
\newcommand{\KeywordTok}[1]{\textcolor[rgb]{0.13,0.29,0.53}{\textbf{#1}}}
\newcommand{\DataTypeTok}[1]{\textcolor[rgb]{0.13,0.29,0.53}{#1}}
\newcommand{\DecValTok}[1]{\textcolor[rgb]{0.00,0.00,0.81}{#1}}
\newcommand{\BaseNTok}[1]{\textcolor[rgb]{0.00,0.00,0.81}{#1}}
\newcommand{\FloatTok}[1]{\textcolor[rgb]{0.00,0.00,0.81}{#1}}
\newcommand{\ConstantTok}[1]{\textcolor[rgb]{0.00,0.00,0.00}{#1}}
\newcommand{\CharTok}[1]{\textcolor[rgb]{0.31,0.60,0.02}{#1}}
\newcommand{\SpecialCharTok}[1]{\textcolor[rgb]{0.00,0.00,0.00}{#1}}
\newcommand{\StringTok}[1]{\textcolor[rgb]{0.31,0.60,0.02}{#1}}
\newcommand{\VerbatimStringTok}[1]{\textcolor[rgb]{0.31,0.60,0.02}{#1}}
\newcommand{\SpecialStringTok}[1]{\textcolor[rgb]{0.31,0.60,0.02}{#1}}
\newcommand{\ImportTok}[1]{#1}
\newcommand{\CommentTok}[1]{\textcolor[rgb]{0.56,0.35,0.01}{\textit{#1}}}
\newcommand{\DocumentationTok}[1]{\textcolor[rgb]{0.56,0.35,0.01}{\textbf{\textit{#1}}}}
\newcommand{\AnnotationTok}[1]{\textcolor[rgb]{0.56,0.35,0.01}{\textbf{\textit{#1}}}}
\newcommand{\CommentVarTok}[1]{\textcolor[rgb]{0.56,0.35,0.01}{\textbf{\textit{#1}}}}
\newcommand{\OtherTok}[1]{\textcolor[rgb]{0.56,0.35,0.01}{#1}}
\newcommand{\FunctionTok}[1]{\textcolor[rgb]{0.00,0.00,0.00}{#1}}
\newcommand{\VariableTok}[1]{\textcolor[rgb]{0.00,0.00,0.00}{#1}}
\newcommand{\ControlFlowTok}[1]{\textcolor[rgb]{0.13,0.29,0.53}{\textbf{#1}}}
\newcommand{\OperatorTok}[1]{\textcolor[rgb]{0.81,0.36,0.00}{\textbf{#1}}}
\newcommand{\BuiltInTok}[1]{#1}
\newcommand{\ExtensionTok}[1]{#1}
\newcommand{\PreprocessorTok}[1]{\textcolor[rgb]{0.56,0.35,0.01}{\textit{#1}}}
\newcommand{\AttributeTok}[1]{\textcolor[rgb]{0.77,0.63,0.00}{#1}}
\newcommand{\RegionMarkerTok}[1]{#1}
\newcommand{\InformationTok}[1]{\textcolor[rgb]{0.56,0.35,0.01}{\textbf{\textit{#1}}}}
\newcommand{\WarningTok}[1]{\textcolor[rgb]{0.56,0.35,0.01}{\textbf{\textit{#1}}}}
\newcommand{\AlertTok}[1]{\textcolor[rgb]{0.94,0.16,0.16}{#1}}
\newcommand{\ErrorTok}[1]{\textcolor[rgb]{0.64,0.00,0.00}{\textbf{#1}}}
\newcommand{\NormalTok}[1]{#1}
\usepackage{longtable,booktabs}
\usepackage{graphicx,grffile}
\makeatletter
\def\maxwidth{\ifdim\Gin@nat@width>\linewidth\linewidth\else\Gin@nat@width\fi}
\def\maxheight{\ifdim\Gin@nat@height>\textheight\textheight\else\Gin@nat@height\fi}
\makeatother
% Scale images if necessary, so that they will not overflow the page
% margins by default, and it is still possible to overwrite the defaults
% using explicit options in \includegraphics[width, height, ...]{}
\setkeys{Gin}{width=\maxwidth,height=\maxheight,keepaspectratio}
\IfFileExists{parskip.sty}{%
\usepackage{parskip}
}{% else
\setlength{\parindent}{0pt}
\setlength{\parskip}{6pt plus 2pt minus 1pt}
}
\setlength{\emergencystretch}{3em}  % prevent overfull lines
\providecommand{\tightlist}{%
  \setlength{\itemsep}{0pt}\setlength{\parskip}{0pt}}
\setcounter{secnumdepth}{0}
% Redefines (sub)paragraphs to behave more like sections
\ifx\paragraph\undefined\else
\let\oldparagraph\paragraph
\renewcommand{\paragraph}[1]{\oldparagraph{#1}\mbox{}}
\fi
\ifx\subparagraph\undefined\else
\let\oldsubparagraph\subparagraph
\renewcommand{\subparagraph}[1]{\oldsubparagraph{#1}\mbox{}}
\fi

%%% Use protect on footnotes to avoid problems with footnotes in titles
\let\rmarkdownfootnote\footnote%
\def\footnote{\protect\rmarkdownfootnote}

%%% Change title format to be more compact
\usepackage{titling}

% Create subtitle command for use in maketitle
\newcommand{\subtitle}[1]{
  \posttitle{
    \begin{center}\large#1\end{center}
    }
}

\setlength{\droptitle}{-2em}

  \title{Midterm Exam - Key}
    \pretitle{\vspace{\droptitle}\centering\huge}
  \posttitle{\par}
  \subtitle{Math 530/630}
  \author{}
    \preauthor{}\postauthor{}
      \predate{\centering\large\emph}
  \postdate{\par}
    \date{Due: 11/20/2018}


\begin{document}
\maketitle

To turn in, please submit 2 files: one is the raw .Rmd file and the
other is a single knitted html(or PDF) file. For the multiple choice,
please clearly indicate the problem number and your answer on separate
lines as in:

\begin{enumerate}
\def\labelenumi{\arabic{enumi}.}
\tightlist
\item
  a
\item
  b
\item
  c
\end{enumerate}

\section{Grading:}\label{grading}

\begin{itemize}
\tightlist
\item
  Free Answer questions are each worth 4 points.
\item
  Multiple Choice are each worth 2 points.
\end{itemize}

You may include R chunks for the multiple choice problems if you like,
but partial credit will not be given.

\section{Key Terms You Should Know}\label{key-terms-you-should-know}

\begin{quote}
\textbf{Alpha-level (\(\alpha\))}: probability of making a Type I error;
that is, rejecting the null hypothesis when it is in fact true (e.g., a
test indicates that a man is pregnant)
\end{quote}

\begin{quote}
\textbf{Alternative Hypothesis} (\(H_A\) or \(H_1\)): the prediction
that there will be an effect.
\end{quote}

\begin{quote}
\textbf{Critical Value(s)}: the value of a given test statistic that
corresponds to a rejection point- the point at which you determine to
reject the null hypothesis. The critical value defines the boundary of
the rejection region for \(H_0\). This value depends on the significance
level, \(\alpha\), and whether the test is one-sided or two-sided.
\end{quote}

\begin{quote}
\textbf{Null Hypothesis} (\(H_0\)): the reverse of the alternative
hypothesis that your prediction is wrong and the predicted effect does
not exist.
\end{quote}

\newpage

\begin{enumerate}
\def\labelenumi{\arabic{enumi}.}
\item
  Given the probability density function (below) of the random variable
  \(X\):

  \begin{enumerate}
  \def\labelenumii{\alph{enumii}.}
  \tightlist
  \item
    find \(c\)
  \item
    find the cumulative distribution function \(F(x)\).
  \item
    Compute \(P(1 < X < 3)\)
  \end{enumerate}
\end{enumerate}

\[f(x) = \begin{cases} 
  cx, & 0 < x < 4 \\
  0, & \textrm{otherwise}
  \end{cases}\]

Answers

\begin{enumerate}
\def\labelenumi{\alph{enumi})}
\item
  \[
  \begin{aligned}
  \int_{0}^{4} cx dx &= 1 \\
    8c &= 1 \\
    c &= \frac{1}{8}
  \end{aligned}
  \]
\item
  As \(\int \frac{1}{8} x dx\) is \(\frac{x^{2}}{16}\),
\end{enumerate}

\begin{longtable}[]{@{}ll@{}}
\toprule
\(x\) & \(F(X)\)\tabularnewline
\midrule
\endhead
\(x<0\) & 0\tabularnewline
\(0 \leq x \leq 4\) & \(\frac{x^{2}}{16}\)\tabularnewline
\(x>4\) & 1\tabularnewline
\bottomrule
\end{longtable}

\begin{enumerate}
\def\labelenumi{\alph{enumi})}
\setcounter{enumi}{2}
\tightlist
\item
  \[
  \begin{aligned}
  P(1 < x < 3) &= F(3) - F(1) \\
    &= \frac{9}{16} - \frac{1}{16} \\
    &= \frac{1}{2}
  \end{aligned}
  \]
\end{enumerate}

\newpage

\begin{enumerate}
\def\labelenumi{\arabic{enumi}.}
\setcounter{enumi}{1}
\item
  A friend claims that she has drawn a random sample of size 30 from the
  exponential distribution with \(\lambda = \frac{1}{10}\). The mean of
  her sample is 12.

  \begin{enumerate}
  \def\labelenumii{\alph{enumii}.}
  \tightlist
  \item
    What is the expected value of a sample mean? (hint:
    \url{https://en.wikipedia.org/wiki/Exponential_distribution\#Mean.2C_variance.2C_moments_and_median})
  \item
    Run a simulation by drawing 1000 random samples, each of size 30
    from Exp(1/10), and then compute the mean. What proportion of the
    sample means are as large or larger than 12?
  \item
    Is a mean of 12 unusual for a sample of size 30 from Exp(1/10)?
  \end{enumerate}
\end{enumerate}

Answers

\begin{enumerate}
\def\labelenumi{(\alph{enumi})}
\item
  \begin{enumerate}
  \def\labelenumii{\arabic{enumii}.}
  \setcounter{enumii}{9}
  \item
  \end{enumerate}
\item
  set.seed(0) my.means\textless{}-numeric(1000) for (i in 1:1000) \{
  x\textless{}-rexp(30, 1/10) my.means{[}i{]}\textless{}-mean(x) \}
  sum(my.means \textgreater{}= 12)/1000 \# .132
\item
  No.
\end{enumerate}

\begin{enumerate}
\def\labelenumi{\arabic{enumi}.}
\setcounter{enumi}{2}
\item
  A researcher believes that in recent years women have been getting
  taller. She knows that 10 years ago the average height of young adult
  women living in her city was 63 inches (\(\sigma = 3\)). She randomly
  samples eight young adult women currently residing in her city and
  measures their heights. The following data are obtained (Height, in
  inches): 64, 66, 68, 60, 62, 65, 66, 63

  \begin{enumerate}
  \def\labelenumii{\alph{enumii}.}
  \tightlist
  \item
    What is the alternative hypothesis? (In evaluating this experiment,
    assume a non-directional hypothesis is appropriate because there are
    insufficient theoretical and empirical bases to warrant a
    directional hypothesis.)
  \item
    What is the null hypothesis?
  \item
    Using the generic test statistic formula given in class
    (\(\frac{\hat{\theta}-{\theta_0}}{SE_{\theta_0}}\)), what is the
    appropriate numerator?
  \item
    Using the generic test statistic formula given in class
    (\(\frac{\hat{\theta}-{\theta_0}}{SE_{\theta_0}}\)), what is the
    appropriate denominator?
  \item
    Using \(\alpha = 0.01\), what is/are the critical value(s) of the
    test statistic?
  \item
    Using \(\alpha = 0.01\), what is the conclusion? (i.e.~do we reject
    or fail to reject the null hypothesis, and why?)
  \item
    Looking at the actual data, how does the sample standard deviation
    compare to that of the population?
  \end{enumerate}
\end{enumerate}

Answers

\begin{enumerate}
\def\labelenumi{\alph{enumi})}
\item
  In recent years the height of women has been changing. Therefore the
  sample with \(\bar{X_{obt}} = 64.25\) is not a random sample from a
  population where \(\mu \neq 63\)
\item
  It is reasonable to consider the sample with \(\bar{X_{obt}} = 64.25\)
  is a random sample from a population where \(\mu = 63\)
\item
  single sample t-test on means
\item
  S = 2.550, \(t_{obt}\) = 1.39
\item
  df = n-1 = 7, \(t_{crit} = \pm 3.499\)
\item
  since \textbar{} \(t_{obt}\) \textbar{} \textless{} \textbar{}
  \(t_{crit}\) \textbar{} , we retain the null hypothesis.
\item
  (61.1, 67.4)
\end{enumerate}

\newpage

\begin{enumerate}
\def\labelenumi{\arabic{enumi}.}
\setcounter{enumi}{3}
\item
  The mean height reported by men on the dating website OKCupid is
  approximately 5 feet 11 inches. For men living in the US, heights are
  normally distributed with a mean of 5 feet 9 inches, \(\sigma\)=3
  inches. Answer the following:

  \begin{enumerate}
  \def\labelenumii{\alph{enumii}.}
  \tightlist
  \item
    State the null and alternative hypotheses.
  \item
    What is/are the critical value(s) of that test statistic (assume
    \(\alpha = .05\), 2-tailed)? That is, for which values of the test
    statistic would you reject the null hypothesis?
  \end{enumerate}
\end{enumerate}

Answers

\begin{enumerate}
\def\labelenumi{\alph{enumi})}
\item
\end{enumerate}

\begin{itemize}
\tightlist
\item
  H0 = OKCupid and US men have the same height
\item
  H1 = OKCupid and US men have different heights
\end{itemize}

\begin{enumerate}
\def\labelenumi{\alph{enumi})}
\setcounter{enumi}{1}
\tightlist
\item
  ll \textless{}- qnorm(.025,69,3) ul \textless{}- qnorm(.975,69,3)
  cbind(ll,ul)
\end{enumerate}

\begin{enumerate}
\def\labelenumi{\arabic{enumi}.}
\setcounter{enumi}{4}
\tightlist
\item
  What are the critical values of a t-distributed random variable
  (\texttt{?TDist}) for each of the following values of \emph{N} and
  \(\alpha\) using nondirectional hypotheses (assume testing of means)?
\end{enumerate}

\begin{quote}
\begin{enumerate}
\def\labelenumi{\alph{enumi}.}
\tightlist
\item
  \emph{N}=12; \(\alpha\)=.05
\item
  \emph{N}=20; \(\alpha\)=.01
\item
  \emph{N}=2; \(\alpha\)=.05
\end{enumerate}
\end{quote}

\begin{quote}
What are the critical values of a t-distributed random variable for each
of the following values of \emph{N} and \(\alpha\) using a directional
hypothesis in the upper tail (assume testing of means)?
\end{quote}

\begin{quote}
\begin{enumerate}
\def\labelenumi{\alph{enumi}.}
\setcounter{enumi}{3}
\tightlist
\item
  \emph{N}=8; \(\alpha\)=.05
\item
  \emph{N}=15; \(\alpha\)=.01
\item
  \emph{N}=51; \(\alpha\)=.025
\end{enumerate}
\end{quote}

Answers

\begin{Shaded}
\begin{Highlighting}[]
\CommentTok{# a)}
\KeywordTok{cbind}\NormalTok{(}\KeywordTok{qt}\NormalTok{(.}\DecValTok{025}\NormalTok{,}\DecValTok{11}\NormalTok{), }\KeywordTok{qt}\NormalTok{(.}\DecValTok{975}\NormalTok{,}\DecValTok{11}\NormalTok{))}
\end{Highlighting}
\end{Shaded}

\begin{verbatim}
##           [,1]     [,2]
## [1,] -2.200985 2.200985
\end{verbatim}

\begin{Shaded}
\begin{Highlighting}[]
\CommentTok{# b)}
\KeywordTok{cbind}\NormalTok{(}\KeywordTok{qt}\NormalTok{(.}\DecValTok{005}\NormalTok{,}\DecValTok{19}\NormalTok{), }\KeywordTok{qt}\NormalTok{(.}\DecValTok{995}\NormalTok{,}\DecValTok{19}\NormalTok{))}
\end{Highlighting}
\end{Shaded}

\begin{verbatim}
##           [,1]     [,2]
## [1,] -2.860935 2.860935
\end{verbatim}

\begin{Shaded}
\begin{Highlighting}[]
\CommentTok{# c)}
\KeywordTok{cbind}\NormalTok{(}\KeywordTok{qt}\NormalTok{(.}\DecValTok{025}\NormalTok{,}\DecValTok{1}\NormalTok{), }\KeywordTok{qt}\NormalTok{(.}\DecValTok{975}\NormalTok{,}\DecValTok{1}\NormalTok{))}
\end{Highlighting}
\end{Shaded}

\begin{verbatim}
##          [,1]    [,2]
## [1,] -12.7062 12.7062
\end{verbatim}

\begin{Shaded}
\begin{Highlighting}[]
\CommentTok{# d)}
\KeywordTok{qt}\NormalTok{(.}\DecValTok{95}\NormalTok{,}\DecValTok{7}\NormalTok{)}
\end{Highlighting}
\end{Shaded}

\begin{verbatim}
## [1] 1.894579
\end{verbatim}

\begin{Shaded}
\begin{Highlighting}[]
\CommentTok{# e)}
\KeywordTok{qt}\NormalTok{(.}\DecValTok{99}\NormalTok{,}\DecValTok{14}\NormalTok{)}
\end{Highlighting}
\end{Shaded}

\begin{verbatim}
## [1] 2.624494
\end{verbatim}

\begin{Shaded}
\begin{Highlighting}[]
\CommentTok{# f)}
\KeywordTok{qt}\NormalTok{(.}\DecValTok{975}\NormalTok{,}\DecValTok{50}\NormalTok{)}
\end{Highlighting}
\end{Shaded}

\begin{verbatim}
## [1] 2.008559
\end{verbatim}

\section{Multiple-Choice Questions}\label{multiple-choice-questions}

\begin{enumerate}
\def\labelenumi{\arabic{enumi}.}
\item
  Suppose you have 10 numbers and have computed the mean to be 8.0. You
  then discover that the last number in the data was entered
  incorrectly. It was entered as 8.0 when it should have been 4.0. If
  you replace the incorrect value (8.0) with the correct one (4.0), and
  recompute the mean, you will obtain a new mean of:

  \begin{enumerate}
  \def\labelenumii{\alph{enumii}.}
  \tightlist
  \item
    It is impossible to determine
  \item
    6.6
  \item
    8.6
  \end{enumerate}
\end{enumerate}

\begin{itemize}
\item
  \begin{enumerate}
  \def\labelenumi{\alph{enumi}.}
  \setcounter{enumi}{3}
  \tightlist
  \item
    7.6
  \end{enumerate}
\end{itemize}

Answer: d

\begin{enumerate}
\def\labelenumi{\arabic{enumi}.}
\setcounter{enumi}{1}
\tightlist
\item
  You have a set of data that have a mean of 50 and a standard deviation
  of 12. You wish them to have a mean of 65 and a standard deviation of
  10, while retaining the shape of the distribution. What values of
  \emph{a} and \emph{b} in the linear transformation formula \emph{Y =
  aX + b} will produce a new set of data with the desired mean and
  standard deviation?
\end{enumerate}

\begin{itemize}
\item
  \begin{enumerate}
  \def\labelenumi{\alph{enumi}.}
  \item
    \emph{a} = 0.833, \emph{b} = 23.3
  \item
    \emph{a} = 1.83, \emph{b} = -23.3
  \item
    \emph{a} = 1.67, \emph{b} = 23.3
  \item
    \emph{a} = 0.833, \emph{b} = 46.7
  \item
    \emph{a} = 1.2, \emph{b} = 22.3
  \end{enumerate}
\end{itemize}

Answer: a

\begin{enumerate}
\def\labelenumi{\arabic{enumi}.}
\setcounter{enumi}{2}
\item
  Jane had a z-score of 1.75 on her statistics midterm. If the class
  mean is 65.0, and the class standard deviation is 12.0, what was
  Jane's raw score?

  \begin{enumerate}
  \def\labelenumii{\alph{enumii}.}
  \tightlist
  \item
    It is impossible to determine
  \item
    66
  \end{enumerate}
\end{enumerate}

\begin{itemize}
\item
  \begin{enumerate}
  \def\labelenumi{\alph{enumi}.}
  \setcounter{enumi}{2}
  \tightlist
  \item
    86
  \item
    76
  \end{enumerate}
\end{itemize}

Answer: c

\begin{enumerate}
\def\labelenumi{\arabic{enumi}.}
\setcounter{enumi}{3}
\tightlist
\item
  IQ scores have a distribution that is approximately normal in shape,
  with a mean of 100 and a standard deviation of 15 in the general
  population. Assuming the normal distribution is a good approximation,
  what proportion of the general population has IQ scores between 79.0
  and 109.0?
\end{enumerate}

\begin{itemize}
\item
  \begin{enumerate}
  \def\labelenumi{\alph{enumi}.}
  \item
    0.645
  \item
    0.745
  \item
    0.545
  \item
    0.709
  \item
    None of the above answers are correct
  \end{enumerate}
\end{itemize}

Answer: a

\begin{Shaded}
\begin{Highlighting}[]
\KeywordTok{pnorm}\NormalTok{(}\DecValTok{109}\NormalTok{, }\DecValTok{100}\NormalTok{, }\DecValTok{15}\NormalTok{) }\OperatorTok{-}\StringTok{ }\KeywordTok{pnorm}\NormalTok{(}\DecValTok{79}\NormalTok{, }\DecValTok{100}\NormalTok{, }\DecValTok{15}\NormalTok{)}
\end{Highlighting}
\end{Shaded}

\begin{verbatim}
## [1] 0.6449902
\end{verbatim}

\begin{enumerate}
\def\labelenumi{\arabic{enumi}.}
\setcounter{enumi}{4}
\tightlist
\item
  You have 10 numbers with a sample mean of 9.0 and a sample variance of
  11.0. You discover that the last number in the list was recorded as
  8.0 when it should have been recorded as 12.0. If you correct your
  error and correctly recompute the sample variance, what value will you
  obtain?
\end{enumerate}

\begin{itemize}
\item
  \begin{enumerate}
  \def\labelenumi{\alph{enumi}.}
  \item
    11.71
  \item
    11.0
  \item
    10.54
  \item
    13.01
  \item
    None of the above answers are correct
  \end{enumerate}
\end{itemize}

Answer: a

\begin{enumerate}
\def\labelenumi{\arabic{enumi}.}
\setcounter{enumi}{5}
\item
  If \(\alpha\) = 0.01, then the probability of a correct acceptance of
  a true statistical null hypothesis is:

  \begin{enumerate}
  \def\labelenumii{\alph{enumii}.}
  \tightlist
  \item
    0.0001
  \item
    \(\beta\)
  \end{enumerate}
\end{enumerate}

\begin{itemize}
\item
  \begin{enumerate}
  \def\labelenumi{\alph{enumi}.}
  \setcounter{enumi}{2}
  \tightlist
  \item
    0.99
  \item
    0
  \item
    It cannot be determined from the information provided
  \end{enumerate}
\end{itemize}

Answer: c

\begin{enumerate}
\def\labelenumi{\arabic{enumi}.}
\setcounter{enumi}{6}
\item
  Given the following probability distribution for the random variable
  \emph{X}, the variance of \emph{X} is:

  \begin{enumerate}
  \def\labelenumii{\alph{enumii}.}
  \tightlist
  \item
    3.5563
  \item
    3.7524
  \item
    2.95
  \item
    1.575
  \end{enumerate}
\end{enumerate}

\begin{itemize}
\item
  \begin{enumerate}
  \def\labelenumi{\alph{enumi}.}
  \setcounter{enumi}{4}
  \tightlist
  \item
    1.8475
  \end{enumerate}
\end{itemize}

\begin{longtable}[]{@{}cc@{}}
\toprule
\emph{x} & \(P_X(x)\)\tabularnewline
\midrule
\endhead
1 & 0.1\tabularnewline
2 & 0.15\tabularnewline
3 & 0.2\tabularnewline
4 & 0.2\tabularnewline
5 & 0.35\tabularnewline
\bottomrule
\end{longtable}

Answer: e

\begin{enumerate}
\def\labelenumi{\arabic{enumi}.}
\setcounter{enumi}{7}
\item
  Given the following probability distribution for the random variable
  \emph{X}, the expected value of \emph{X} is:

  \begin{enumerate}
  \def\labelenumii{\alph{enumii}.}
  \tightlist
  \item
    2.48
  \item
    3.68
  \item
    3.58
  \end{enumerate}
\end{enumerate}

\begin{itemize}
\item
  \begin{enumerate}
  \def\labelenumi{\alph{enumi}.}
  \setcounter{enumi}{3}
  \tightlist
  \item
    3.48
  \end{enumerate}
\end{itemize}

\begin{longtable}[]{@{}cc@{}}
\toprule
\emph{x} & \(P_X(x)\)\tabularnewline
\midrule
\endhead
1 & 0.1\tabularnewline
2 & 0.2\tabularnewline
3 & 0.2\tabularnewline
4 & 0.12\tabularnewline
5 & 0.38\tabularnewline
\bottomrule
\end{longtable}

Answer: d

\begin{enumerate}
\def\labelenumi{\arabic{enumi}.}
\setcounter{enumi}{8}
\item
  The sampling distribution of the sample mean based on \emph{N} iid
  observations

  \begin{enumerate}
  \def\labelenumii{\alph{enumii}.}
  \tightlist
  \item
    converges asymptotically to a normal distribution in shape under the
    conditions of the Central Limit Theorem
  \item
    has a variance of \(\sigma^2/N\) for any population distribution
  \item
    is always exactly normal, for any sample size, when the population
    distribution is normal
  \end{enumerate}
\end{enumerate}

\begin{itemize}
\item
  \begin{enumerate}
  \def\labelenumi{\alph{enumi}.}
  \setcounter{enumi}{3}
  \tightlist
  \item
    all of the above answers are correct
  \end{enumerate}
\end{itemize}

Answer: d

\begin{enumerate}
\def\labelenumi{\arabic{enumi}.}
\setcounter{enumi}{9}
\item
  If one draws all possible samples for various values of \emph{N} from
  the same population of raw scores, as \emph{N} increases:

  \begin{enumerate}
  \def\labelenumii{\alph{enumii}.}
  \tightlist
  \item
    The standard error of the mean increases
  \item
    The standard error of the mean stays the same
  \end{enumerate}
\end{enumerate}

\begin{itemize}
\item
  \begin{enumerate}
  \def\labelenumi{\alph{enumi}.}
  \setcounter{enumi}{2}
  \tightlist
  \item
    The standard error of the mean decreases
  \item
    The standard error of the mean cannot be calculated
  \end{enumerate}
\end{itemize}

Answer: c

\begin{enumerate}
\def\labelenumi{\arabic{enumi}.}
\setcounter{enumi}{10}
\item
  If one draws all possible samples for various values of \emph{N} from
  the same population of raw scores, as \emph{N} increases:

  \begin{enumerate}
  \def\labelenumii{\alph{enumii}.}
  \tightlist
  \item
    The mean of the sampling distribution of the mean increases
  \end{enumerate}
\end{enumerate}

\begin{itemize}
\item
  \begin{enumerate}
  \def\labelenumi{\alph{enumi}.}
  \setcounter{enumi}{1}
  \tightlist
  \item
    The mean of the sampling distribution of the mean stays the same
  \item
    The mean of the sampling distribution of the mean decreases
  \item
    None of the above
  \end{enumerate}
\end{itemize}

Answer: b

\begin{enumerate}
\def\labelenumi{\arabic{enumi}.}
\setcounter{enumi}{11}
\item
  In cases where \emph{N} \textgreater{} 1, the relationship between the
  raw score population standard deviation and the standard error is:

  \begin{enumerate}
  \def\labelenumii{\alph{enumii}.}
  \tightlist
  \item
    The standard error is greater than the population standard deviation
  \end{enumerate}
\end{enumerate}

\begin{itemize}
\item
  \begin{enumerate}
  \def\labelenumi{\alph{enumi}.}
  \setcounter{enumi}{1}
  \tightlist
  \item
    The standard error is less than the population standard deviation
  \item
    The standard error equals the population standard deviation
  \item
    The standard error is the population standard deviation
  \end{enumerate}
\end{itemize}

Answer: b

\begin{enumerate}
\def\labelenumi{\arabic{enumi}.}
\setcounter{enumi}{12}
\item
  The variance can be thought of as:

  \begin{enumerate}
  \def\labelenumii{\alph{enumii}.}
  \tightlist
  \item
    half the range
  \item
    the sum of squared deviations from the mean
  \item
    the average deviation
  \end{enumerate}
\end{enumerate}

\begin{itemize}
\item
  \begin{enumerate}
  \def\labelenumi{\alph{enumi}.}
  \setcounter{enumi}{3}
  \tightlist
  \item
    the average squared deviation from the mean
  \end{enumerate}
\end{itemize}

Answer: d

\begin{enumerate}
\def\labelenumi{\arabic{enumi}.}
\setcounter{enumi}{13}
\item
  What would happen to the mean of a distribution of scores if the
  number 10 is added to each score?

  \begin{enumerate}
  \def\labelenumii{\alph{enumii}.}
  \tightlist
  \item
    It would stay the same
  \end{enumerate}
\end{enumerate}

\begin{itemize}
\item
  \begin{enumerate}
  \def\labelenumi{\alph{enumi}.}
  \setcounter{enumi}{1}
  \tightlist
  \item
    It increases by 10
  \item
    It will become 10 times as large
  \item
    It will increase, but the amount depends on the shape of the
    distribution
  \end{enumerate}
\end{itemize}

Answer: b

\begin{enumerate}
\def\labelenumi{\arabic{enumi}.}
\setcounter{enumi}{14}
\tightlist
\item
  What would happen to the standard deviation of a distribution of
  scores if the number 10 is added to each score?
\end{enumerate}

\begin{itemize}
\item
  \begin{enumerate}
  \def\labelenumi{\alph{enumi}.}
  \item
    It would stay the same
  \item
    It increases by 10
  \item
    It will become 10 times as large
  \item
    It will increase, but the amount depends on the shape of the
    distribution
  \end{enumerate}
\end{itemize}

Answer: a

\begin{enumerate}
\def\labelenumi{\arabic{enumi}.}
\setcounter{enumi}{15}
\item
  What would happen to the mean of a distribution of scores if each
  score is multiplied by 2?

  \begin{enumerate}
  \def\labelenumii{\alph{enumii}.}
  \tightlist
  \item
    It would stay the same
  \item
    It increases by 2
  \end{enumerate}
\end{enumerate}

\begin{itemize}
\item
  \begin{enumerate}
  \def\labelenumi{\alph{enumi}.}
  \setcounter{enumi}{2}
  \tightlist
  \item
    It will become twice as large
  \item
    It will increase, but the amount depends on the shape of the
    distribution
  \end{enumerate}
\end{itemize}

Answer: c

\begin{enumerate}
\def\labelenumi{\arabic{enumi}.}
\setcounter{enumi}{16}
\item
  What would happen to the variance of a distribution of scores if each
  score is multiplied by 2?

  \begin{enumerate}
  \def\labelenumii{\alph{enumii}.}
  \tightlist
  \item
    It would stay the same
  \item
    It will become twice as large
  \end{enumerate}
\end{enumerate}

\begin{itemize}
\item
  \begin{enumerate}
  \def\labelenumi{\alph{enumi}.}
  \setcounter{enumi}{2}
  \tightlist
  \item
    It will become four times as large
  \item
    It will increase, but the amount depends on the shape of the
    distribution
  \end{enumerate}
\end{itemize}

Answer: c

\begin{enumerate}
\def\labelenumi{\arabic{enumi}.}
\setcounter{enumi}{17}
\item
  A researcher has data for two variables, \emph{x} and \emph{y}. First,
  she converts both variables to z-scores with a mean of 0 and standard
  deviation of 1, and calls them \(z_x\) and \(z_y\). Next, she takes
  the mean of both z-scores and calls that new variable \(ave_z\). What
  is the mean and standard deviation of the \(ave_z\) variable?

  \begin{enumerate}
  \def\labelenumii{\alph{enumii}.}
  \tightlist
  \item
    mean = 0; standard deviation = 1
  \item
    mean = 0; standard deviation approximately 1
  \end{enumerate}
\end{enumerate}

\begin{itemize}
\item
  \begin{enumerate}
  \def\labelenumi{\alph{enumi}.}
  \setcounter{enumi}{2}
  \tightlist
  \item
    mean = 0; standard deviation \textgreater{} 1
  \item
    Cannot be determined from the information given
  \end{enumerate}
\end{itemize}

Answer: c

\begin{Shaded}
\begin{Highlighting}[]
\KeywordTok{library}\NormalTok{(dplyr)}
\KeywordTok{options}\NormalTok{(}\DataTypeTok{scipen=}\DecValTok{999}\NormalTok{)}
\NormalTok{x <-}\StringTok{ }\KeywordTok{c}\NormalTok{(}\DecValTok{0}\NormalTok{, }\OperatorTok{-}\DecValTok{1}\NormalTok{, }\DecValTok{1}\NormalTok{, }\OperatorTok{-}\DecValTok{5}\NormalTok{, }\DecValTok{10}\NormalTok{)}
\NormalTok{y <-}\StringTok{ }\KeywordTok{c}\NormalTok{(}\OperatorTok{-}\DecValTok{2}\NormalTok{, }\OperatorTok{-}\DecValTok{3}\NormalTok{, }\DecValTok{5}\NormalTok{, }\OperatorTok{-}\DecValTok{7}\NormalTok{, }\DecValTok{7}\NormalTok{)}
\NormalTok{z_x <-}\StringTok{ }\KeywordTok{scale}\NormalTok{(x)}
\NormalTok{z_y <-}\StringTok{ }\KeywordTok{scale}\NormalTok{(y)}
\NormalTok{my_z <-}\StringTok{ }\KeywordTok{data.frame}\NormalTok{(z_x, z_y)}
\NormalTok{my_z }\OperatorTok\StringTok{ }
\StringTok{  }\KeywordTok{mutate}\NormalTok{(}\DataTypeTok{ave_z =}\NormalTok{ z_x }\OperatorTok{+}\StringTok{ }\NormalTok{z_y }\OperatorTok{/}\StringTok{ }\DecValTok{2}\NormalTok{) }\OperatorTok\StringTok{ }
\StringTok{  }\KeywordTok{summarise_each}\NormalTok{(}\KeywordTok{funs}\NormalTok{(mean, sd))}
\end{Highlighting}
\end{Shaded}

\begin{verbatim}
##                     z_x_mean z_y_mean                 ave_z_mean z_x_sd
## 1 -0.00000000000000003885781        0 -0.00000000000000002220446      1
##   z_y_sd ave_z_sd
## 1      1 1.458513
\end{verbatim}

\begin{enumerate}
\def\labelenumi{\arabic{enumi}.}
\setcounter{enumi}{18}
\item
  The sampling distribution and standard error of a statistic can be
  calculated by:

  \begin{enumerate}
  \def\labelenumii{\alph{enumii}.}
  \tightlist
  \item
    exhaustive and exact calculations where formula solutions are
    possible
  \item
    simulation
  \item
    formula approximations
  \item
    repeatedly taking random samples of a given size from a population
  \end{enumerate}
\end{enumerate}

\begin{itemize}
\item
  \begin{enumerate}
  \def\labelenumi{\alph{enumi}.}
  \setcounter{enumi}{4}
  \tightlist
  \item
    all of the above
  \end{enumerate}
\end{itemize}

Answer: e

\begin{enumerate}
\def\labelenumi{\arabic{enumi}.}
\setcounter{enumi}{19}
\item
  IQ scores have a distribution that is approximately normal in shape,
  with a mean of 100 and a standard deviation of 15. What percentage of
  scores is at or above an IQ of 116?

  \begin{enumerate}
  \def\labelenumii{\alph{enumii}.}
  \tightlist
  \item
    12.464
  \end{enumerate}
\end{enumerate}

\begin{itemize}
\item
  \begin{enumerate}
  \def\labelenumi{\alph{enumi}.}
  \setcounter{enumi}{1}
  \tightlist
  \item
    14.306
  \item
    15.737
  \item
    16.355
  \item
    None of the above answers are correct.
  \end{enumerate}
\end{itemize}

\begin{Shaded}
\begin{Highlighting}[]
\CommentTok{#the answer is b}
\DecValTok{1}\OperatorTok{-}\KeywordTok{pnorm}\NormalTok{(}\DecValTok{116}\NormalTok{,}\DecValTok{100}\NormalTok{,}\DecValTok{15}\NormalTok{)}
\end{Highlighting}
\end{Shaded}

\begin{verbatim}
## [1] 0.1430612
\end{verbatim}

\begin{enumerate}
\def\labelenumi{\arabic{enumi}.}
\setcounter{enumi}{20}
\tightlist
\item
  Suppose you want to test the null hypothesis that \(\mu\)=100 with a
  sample size of \emph{n}=25 and an \(\alpha\)=.05 using a t-statistic,
  which you know follows the Student t-distribution (\texttt{?TDist}).
  What will the critical value(s) for the \emph{t} statistic be? That
  is, for which values of the \emph{t} statistic will you reject the
  null hypothesis?
\end{enumerate}

\begin{itemize}
\item
  \begin{enumerate}
  \def\labelenumi{\alph{enumi}.}
  \item
    \(\leq\) -2.06, \(\geq\) 2.06
  \item
    \(\leq\) -1.71, \(\geq\) 1.71
  \item
    \(\leq\) -2.06, \(\geq\) 1.85
  \item
    \(\leq\) -1.85, \(\geq\) 2.06
  \item
    None of the above answers are correct.
  \end{enumerate}
\end{itemize}

\begin{Shaded}
\begin{Highlighting}[]
\CommentTok{# a}
\NormalTok{ul <-}\StringTok{ }\KeywordTok{qt}\NormalTok{(.}\DecValTok{975}\NormalTok{, }\DecValTok{24}\NormalTok{)}
\NormalTok{ll <-}\StringTok{ }\KeywordTok{qt}\NormalTok{(.}\DecValTok{025}\NormalTok{, }\DecValTok{24}\NormalTok{)}
\KeywordTok{cbind}\NormalTok{(ll,ul)}
\end{Highlighting}
\end{Shaded}

\begin{verbatim}
##             ll       ul
## [1,] -2.063899 2.063899
\end{verbatim}

\begin{enumerate}
\def\labelenumi{\arabic{enumi}.}
\setcounter{enumi}{21}
\tightlist
\item
  Could the sample \emph{X}=\{21,21,21,20,22,20,22\} reasonably have
  been drawn from a normal population with a mean of 20 and standard
  deviation of 1.5 with \(\alpha\)=0.05 (two-tailed)?
\end{enumerate}

\begin{itemize}
\item
  \begin{enumerate}
  \def\labelenumi{\alph{enumi}.}
  \item
    yes
  \item
    no
  \item
    cannot be tested with \emph{z} test
  \item
    insufficient information
  \end{enumerate}
\end{itemize}

Answer: a

\begin{center}\rule{0.5\linewidth}{\linethickness}\end{center}

For the last set of questions, suppose you have four children in a
reading group (Beth, Marianne, Steven, Joel) and you randomly pick one
child to lead the discussion in group each day of a 5-day week.
Furthermore, we define the outcome of each day's selection to be binary:
Steven leads the discussion or he doesn't.

\begin{enumerate}
\def\labelenumi{\arabic{enumi}.}
\setcounter{enumi}{22}
\item
  The number of times Steven leads the discussion in a week would be
  the:

  \begin{enumerate}
  \def\labelenumii{\alph{enumii}.}
  \tightlist
  \item
    probability distribution for this experiment
  \item
    probability of an outcome
  \item
    constant in this experiment
  \end{enumerate}
\end{enumerate}

\begin{itemize}
\item
  \begin{enumerate}
  \def\labelenumi{\alph{enumi}.}
  \setcounter{enumi}{3}
  \tightlist
  \item
    random variable in this experiment
  \end{enumerate}
\end{itemize}

Answer: d

\begin{enumerate}
\def\labelenumi{\arabic{enumi}.}
\setcounter{enumi}{23}
\item
  The probability that Steven leads the discussion all 5 days in a week
  is the:

  \begin{enumerate}
  \def\labelenumii{\alph{enumii}.}
  \tightlist
  \item
    expected value
  \item
    probability distribution
  \end{enumerate}
\end{enumerate}

\begin{itemize}
\item
  \begin{enumerate}
  \def\labelenumi{\alph{enumi}.}
  \setcounter{enumi}{2}
  \tightlist
  \item
    probability of a simple event
  \item
    random variable
  \end{enumerate}
\end{itemize}

Answer: c

\begin{enumerate}
\def\labelenumi{\arabic{enumi}.}
\setcounter{enumi}{24}
\tightlist
\item
  If we could replicate this experiment many, many times, the average
  number of times that Steven leads the discussion in a week would be
  the:
\end{enumerate}

\begin{itemize}
\item
  \begin{enumerate}
  \def\labelenumi{\alph{enumi}.}
  \item
    expected value
  \item
    probability distribution
  \item
    probability of an outcome
  \item
    random variable
  \end{enumerate}
\end{itemize}

Answer: a

\begin{enumerate}
\def\labelenumi{\arabic{enumi}.}
\setcounter{enumi}{25}
\item
  If we found the probability of Steven leading the discussion zero
  times, one time, two times, three times, four times, and five times,
  the set of six probabilities would be the:

  \begin{enumerate}
  \def\labelenumii{\alph{enumii}.}
  \tightlist
  \item
    expected value
  \end{enumerate}
\end{enumerate}

\begin{itemize}
\item
  \begin{enumerate}
  \def\labelenumi{\alph{enumi}.}
  \setcounter{enumi}{1}
  \tightlist
  \item
    probability distribution
  \item
    probability of an outcome
  \item
    random variable
  \end{enumerate}
\end{itemize}

Answer: b

\begin{enumerate}
\def\labelenumi{\arabic{enumi}.}
\setcounter{enumi}{26}
\item
  Monday's selection of a discussion leader could be considered a
  \_\_\_\_, while the selections of discussion leaders for the week
  constitute a \_\_\_\_:

  \begin{enumerate}
  \def\labelenumii{\alph{enumii}.}
  \tightlist
  \item
    binomial experiment, bernoulli trial
  \end{enumerate}
\end{enumerate}

\begin{itemize}
\item
  \begin{enumerate}
  \def\labelenumi{\alph{enumi}.}
  \setcounter{enumi}{1}
  \tightlist
  \item
    bernoulli trial, binomial experiment
  \item
    neither
  \end{enumerate}
\end{itemize}

Answer: b

\begin{enumerate}
\def\labelenumi{\arabic{enumi}.}
\setcounter{enumi}{27}
\item
  What is the probability that Steven would be selected all 5 days of
  the week?

  \begin{enumerate}
  \def\labelenumii{\alph{enumii}.}
  \tightlist
  \item
    .00000000000000
  \end{enumerate}
\end{enumerate}

\begin{itemize}
\item
  \begin{enumerate}
  \def\labelenumi{\alph{enumi}.}
  \setcounter{enumi}{1}
  \tightlist
  \item
    .0009765625
  \item
    .0039065500
  \item
    .25
  \end{enumerate}
\end{itemize}

Answer: b

\begin{Shaded}
\begin{Highlighting}[]
\NormalTok{.}\DecValTok{25}\OperatorTok{^}\DecValTok{5} \CommentTok{#or}
\end{Highlighting}
\end{Shaded}

\begin{verbatim}
## [1] 0.0009765625
\end{verbatim}

\begin{Shaded}
\begin{Highlighting}[]
\KeywordTok{dbinom}\NormalTok{(}\DecValTok{5}\NormalTok{, }\DecValTok{5}\NormalTok{, .}\DecValTok{25}\NormalTok{) }\CommentTok{#or}
\end{Highlighting}
\end{Shaded}

\begin{verbatim}
## [1] 0.0009765625
\end{verbatim}

\begin{Shaded}
\begin{Highlighting}[]
\DecValTok{1} \OperatorTok{-}\StringTok{ }\KeywordTok{pbinom}\NormalTok{(}\DecValTok{4}\NormalTok{, }\DecValTok{5}\NormalTok{, .}\DecValTok{25}\NormalTok{) }\CommentTok{#or}
\end{Highlighting}
\end{Shaded}

\begin{verbatim}
## [1] 0.0009765625
\end{verbatim}

\begin{Shaded}
\begin{Highlighting}[]
\KeywordTok{pbinom}\NormalTok{(}\DecValTok{4}\NormalTok{, }\DecValTok{5}\NormalTok{, .}\DecValTok{25}\NormalTok{, }\DataTypeTok{lower.tail =} \OtherTok{FALSE}\NormalTok{)}
\end{Highlighting}
\end{Shaded}

\begin{verbatim}
## [1] 0.0009765625
\end{verbatim}

\begin{enumerate}
\def\labelenumi{\arabic{enumi}.}
\setcounter{enumi}{28}
\item
  What is the expected number of times that Steven would be selected?

  \begin{enumerate}
  \def\labelenumii{\alph{enumii}.}
  \tightlist
  \item
    0.00
  \item
    1.00
  \end{enumerate}
\end{enumerate}

\begin{itemize}
\item
  \begin{enumerate}
  \def\labelenumi{\alph{enumi}.}
  \setcounter{enumi}{2}
  \tightlist
  \item
    1.25
  \item
    2.50
  \end{enumerate}
\end{itemize}

Answer: c

\begin{Shaded}
\begin{Highlighting}[]
\NormalTok{n <-}\StringTok{ }\DecValTok{5}
\NormalTok{p <-}\StringTok{ }\NormalTok{.}\DecValTok{25}
\NormalTok{n}\OperatorTok{*}\NormalTok{p }\CommentTok{# definition of expected value for a binomial rv}
\end{Highlighting}
\end{Shaded}

\begin{verbatim}
## [1] 1.25
\end{verbatim}

\begin{center}\rule{0.5\linewidth}{\linethickness}\end{center}

\begin{enumerate}
\def\labelenumi{\arabic{enumi}.}
\setcounter{enumi}{29}
\item
  Using the following code, run a permutation test with the hypothesis
  that mean Hotwing consumption is different for men than for women
  (assume \(\alpha = .05\), 2-tailed). Does the data support your
  hypothesis, and what is the associated p-value?

  \begin{enumerate}
  \def\labelenumii{\alph{enumii}.}
  \tightlist
  \item
    The data supports \(H_A\), p = 0.00078
  \end{enumerate}
\end{enumerate}

\begin{itemize}
\item
  \begin{enumerate}
  \def\labelenumi{\alph{enumi}.}
  \setcounter{enumi}{1}
  \tightlist
  \item
    The data supports \(H_A\), p = 0.00156
  \item
    We fail to reject \(H_0\), p = 0.156
  \item
    We fail to reject \(H_0\), p = 0.780
  \end{enumerate}
\end{itemize}

\begin{Shaded}
\begin{Highlighting}[]
\CommentTok{#install.packages("resampledata")}
\KeywordTok{library}\NormalTok{(resampledata)}
\KeywordTok{library}\NormalTok{(dplyr)}
\end{Highlighting}
\end{Shaded}

\begin{Shaded}
\begin{Highlighting}[]
\KeywordTok{glimpse}\NormalTok{(Beerwings)}
\end{Highlighting}
\end{Shaded}

\begin{verbatim}
## Observations: 30
## Variables: 4
## $ ID       <int> 1, 2, 3, 4, 5, 6, 7, 8, 9, 10, 11, 12, 13, 14, 15, 16...
## $ Hotwings <int> 4, 5, 5, 6, 7, 7, 7, 8, 8, 8, 9, 11, 11, 12, 12, 13, ...
## $ Beer     <int> 24, 0, 12, 12, 12, 12, 24, 24, 0, 12, 24, 24, 24, 30,...
## $ Gender   <fct> F, F, F, F, F, F, M, F, M, M, F, F, M, F, F, F, F, M,...
\end{verbatim}

\begin{Shaded}
\begin{Highlighting}[]
\CommentTok{# observed mean difference}
\NormalTok{Beerwings }\OperatorTok\StringTok{ }
\StringTok{  }\KeywordTok{group_by}\NormalTok{(Gender) }\OperatorTok\StringTok{ }
\StringTok{  }\KeywordTok{summarise}\NormalTok{(}\DataTypeTok{mean_wings =} \KeywordTok{mean}\NormalTok{(Hotwings)) }
\end{Highlighting}
\end{Shaded}

\begin{verbatim}
## # A tibble: 2 x 2
##   Gender mean_wings
##   <fct>       <dbl>
## 1 F            9.33
## 2 M           14.5
\end{verbatim}

Run the permutation:

\begin{Shaded}
\begin{Highlighting}[]
\KeywordTok{set.seed}\NormalTok{(}\DecValTok{0}\NormalTok{)}
\NormalTok{B <-}\StringTok{ }\DecValTok{10}\OperatorTok{^}\DecValTok{5}\OperatorTok{-}\DecValTok{1}  \CommentTok{#set number of times to repeat this process}
\NormalTok{result <-}\StringTok{ }\KeywordTok{numeric}\NormalTok{(B) }\CommentTok{# space to save the random differences}
\ControlFlowTok{for}\NormalTok{(i }\ControlFlowTok{in} \DecValTok{1}\OperatorTok{:}\NormalTok{B)\{}
\NormalTok{  index <-}\StringTok{ }\KeywordTok{sample}\NormalTok{(}\DecValTok{30}\NormalTok{, }\DataTypeTok{size=}\DecValTok{15}\NormalTok{, }\DataTypeTok{replace =} \OtherTok{FALSE}\NormalTok{) }\CommentTok{# sample of numbers from 1:30}
\NormalTok{  result[i] <-}\StringTok{ }\KeywordTok{mean}\NormalTok{(Beerwings}\OperatorTok{$}\NormalTok{Hotwings[index]) }\OperatorTok{-}\StringTok{ }\KeywordTok{mean}\NormalTok{(Beerwings}\OperatorTok{$}\NormalTok{Hotwings[}\OperatorTok{-}\NormalTok{index])}
\NormalTok{\}}
\end{Highlighting}
\end{Shaded}

\begin{Shaded}
\begin{Highlighting}[]
\CommentTok{# As calculated.}
\CommentTok{# ...and some commented out alternatives that are equivalent at 4 decimal places :)}
\NormalTok{observed <-}\StringTok{ }\FloatTok{14.5333}\OperatorTok{-}\StringTok{ }\FloatTok{9.3333} 
\CommentTok{#!! Only use the count from the smaller tail, and then double it}
\NormalTok{min_sum <-}\StringTok{ }\KeywordTok{min}\NormalTok{(}\KeywordTok{sum}\NormalTok{(result }\OperatorTok{>=}\StringTok{ }\NormalTok{observed), }\KeywordTok{sum}\NormalTok{(result }\OperatorTok{<=}\StringTok{ }\OperatorTok{-}\NormalTok{observed))}
\CommentTok{#min_sum <- min(sum(result > observed), sum(result < -observed))}
\CommentTok{#Compute P-value (adjusted p-value)}
\NormalTok{min_p <-}\StringTok{ }\KeywordTok{sum}\NormalTok{(min_sum }\OperatorTok{+}\StringTok{ }\DecValTok{1}\NormalTok{)}\OperatorTok{/}\NormalTok{(B }\OperatorTok{+}\StringTok{ }\DecValTok{1}\NormalTok{)}
\CommentTok{#Unadjusted p-value}
\CommentTok{#min_p <- min_sum/B}
\DecValTok{2}\OperatorTok{*}\NormalTok{min_p}
\end{Highlighting}
\end{Shaded}

\begin{verbatim}
## [1] 0.00156
\end{verbatim}

\begin{Shaded}
\begin{Highlighting}[]
\CommentTok{#Using infer}

\NormalTok{observed <-}\StringTok{ }\FloatTok{14.5333}\OperatorTok{-}\StringTok{ }\FloatTok{9.3333} 
\KeywordTok{library}\NormalTok{(infer)}
\NormalTok{null_distn_two_means <-Beerwings }\OperatorTok
\StringTok{  }\KeywordTok{specify}\NormalTok{(Hotwings }\OperatorTok{~}\StringTok{ }\NormalTok{Gender) }\OperatorTok\StringTok{ }
\StringTok{  }\KeywordTok{hypothesize}\NormalTok{(}\DataTypeTok{null =} \StringTok{"independence"}\NormalTok{) }\OperatorTok\StringTok{ }
\StringTok{  }\KeywordTok{generate}\NormalTok{(}\DataTypeTok{reps =} \DecValTok{100000}\NormalTok{) }\OperatorTok\StringTok{ }
\StringTok{  }\KeywordTok{calculate}\NormalTok{(}\DataTypeTok{stat =} \StringTok{"diff in means"}\NormalTok{,}
            \DataTypeTok{order =} \KeywordTok{c}\NormalTok{(}\StringTok{"M"}\NormalTok{, }\StringTok{"F"}\NormalTok{))}
  
\NormalTok{pvalue <-}\StringTok{ }\NormalTok{null_distn_two_means }\OperatorTok\StringTok{ }
\StringTok{  }\KeywordTok{get_pvalue}\NormalTok{(}\DataTypeTok{obs_stat =}\NormalTok{ observed, }\DataTypeTok{direction =} \StringTok{"both"}\NormalTok{)}
\NormalTok{pvalue  }
\end{Highlighting}
\end{Shaded}

\begin{verbatim}
## # A tibble: 1 x 1
##   p_value
##     <dbl>
## 1 0.00147
\end{verbatim}


\end{document}
